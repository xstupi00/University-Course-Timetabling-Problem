\documentclass{PPFIT} % pro psaní v angličtině / for writing in English
% \documentclass[czech]{PPFIT} % pro psaní v češtině / for writing in Czech
%\documentclass[slovak]{PPFIT} % pro psaní ve slovenštině / for writing in Slovak

%--------------------------------------------------------
%--------------------------------------------------------
%	PŘIZPŮSOBENÍ PDF / PDF CUSTOMIZATION
%--------------------------------------------------------

\hypersetup{
	pdftitle={Paper Title},
	pdfauthor={Author},
	pdfkeywords={Keyword1, Keyword2, Keyword3}
}

%--------------------------------------------------------
%--------------------------------------------------------
%  PŘEDBĚŽNÁ vs. KONEČNÁ VERZE / REVIEW vs. FINAL VERSION
%--------------------------------------------------------
%   NECHTE následující řádek zakomentovaný pro PŘEDBĚŽNÉ VERZE
%   Pro KONEČNOU VERZI řádek odkomentujte.
%   LEAVE this line commented out for the REVIEW VERSIONS
%   UNCOMMENT this line to get the FINAL VERSION

\PPFinalCopy


\PPYear{2020/2021}

\ifczechslovak%
    %--------------------------------------------------------
    %--------------------------------------------------------
    %	INFORMACE O ČLÁNKU
    %--------------------------------------------------------    

  \PaperTitle{Jak napsat skvělý článek}
	
	\Authors{Adam Herout*}
	\affiliation{*%
	  \href{mailto:herout@fit.vut.cz}{herout@fit.vut.cz},
	  \textit{Fakulta informačních technologií, Vysoké učení technické v Brně}}

	\Keywords{Klíčové slovo1 --- Klíčové slovo2 --- Klíčové slovo3}
	
	\Supplementary{\href{http://youtu.be/S3msCdn3fNM}{Demonstrační Video} --- \href{http://fit.vut.cz/}{Stáhnutelný Kód}}

    %--------------------------------------------------------
    %--------------------------------------------------------
    %	ABSTRAKT
    %--------------------------------------------------------
    \Abstract{
        O jaký problém se jedná? O jaké téma? Jaký je cíl této práce?
        \phony{Lorem ipsum dolor sit amet, consectetur adipiscing elit. Fusce ullamcorper suscipit euismod. Mauris sed lectus non massa molestie congue. In hac habitasse platea dictumst.}
        %	
        Jak se tento problém řeší a jak je cíle dosaženo (metodika)?
        \phony{Lorem ipsum dolor sit amet, consectetur adipiscing elit. Fusce ullamcorper suscipit euismod. Mauris sed lectus non massa molestie congue. In hac habitasse platea dictumst. Curabitur massa neque, commodo posuere fringilla ut, cursus at dui. Nulla quis purus a justo pellentesque.}
        %	
        Jaké jsou konkrétní výsledky? Jak dobře se daný problém řeší?
        \phony{Lorem ipsum dolor sit amet, consectetur adipiscing elit. Fusce ullamcorper suscipit euismod. Mauris sed lectus non massa molestie congue. In hac habitasse platea dictumst.}
        %	
        Takže co? Jak užitečná je tato práce pro vědu a pro čtenáře?
        \phony{Lorem ipsum dolor sit amet, consectetur adipiscing elit. Fusce ullamcorper suscipit euismod.}
    }
\else%
    %--------------------------------------------------------
    %--------------------------------------------------------
    %	INFORMATION ABOUT THE ARTICLE
    %--------------------------------------------------------
    \PaperTitle{Simulation Tools and Techniques (SNT)\\ University Course Timetabling Problem (UCTP)}
	
	\Authors{Šimon Stupinský,
		\href{mailto:xstupi00@fit.vut.cz}{xstupi00@fit.vut.cz},\\
		\textit{Faculty of Information Technology, Brno University of Technology}}
	\affiliation{*%
		\href{mailto:xstupi00@fit.vut.cz}{xstupi00@fit.vut.cz},
		\textit{Faculty of Information Technology, Brno University of Technology}}

	\Keywords{Keyword1 --- Keyword2 --- Keyword3}
	
	\Supplementary{\href{http://youtu.be/S3msCdn3fNM}{Demonstation video} --- \href{http://fit.vut.cz/}{Downloadable code}}
	
	%--------------------------------------------------------
    %--------------------------------------------------------
    %	ABSTRACT
    %--------------------------------------------------------
    \Abstract{}
\fi

%--------------------------------------------------------
%--------------------------------------------------------
%	TEASER 
%--------------------------------------------------------
\Teaser{
%	\TeaserImage{placeholder.pdf}
%	\TeaserImage{placeholder.pdf}
%	\TeaserImage{placeholder.pdf}
}

%--------------------------------------------------------
%--------------------------------------------------------
%--------------------------------------------------------
%--------------------------------------------------------
\begin{document}
\startdocument
%--------------------------------------------------------
%--------------------------------------------------------
%	OBSAH ČLÁNKU / CONTENT OF THE ARTICLE
%--------------------------------------------------------
\ifczechslovak
    \input{2019-PPFIT-ShortName-text.tex}
\else
    \section{Introduction}
This project describes a~selected \textit{operational research} problem according to the~particular published research paper~\cite{paper}.
A~given problem is defined in Chapter~\ref{sec:problem}.
Approaches to a~solution to the~problem from the~paper are re-implemented,~and experimental results are finally compared with the~results in the~paper and with the~student results from the~previous years.

The~rest of this documentation is organised as follows.
Chapter~\ref{sec:method} presents an~approach to the~solution to the~problem,~and there are also described appropriate methods.
Re-implementation of the method is then depicted in Chapter~\ref{sec:implementation}.
Experimental evaluation is discussed in Chapter~\ref{sec:experiments}.
Finally,~Chapter~\ref{sec:conclusion} concludes this documentation and achieved project results.

\section{University Course Timetabling Problem} \label{sec:problem}
\textit{University Course Timetabling Problem} (\textit{UCTP}) is known as an~\emph{NP-hard} problem.
It is defined as a~set of rooms,~events in the~schedule,~students and room features,~the~completion of which individual courses require.
The UCTP solution is then assigned free room and item in the~time-schedule for each of a~given number of courses for meeting all the~specified features and conditions.
In more precise,~an~UCTP is an~optimisation problem that deals with the~scheduling process. 
The~aim is to find such a~distribution of the course into classrooms and schedule times,~which does not conflict with \textit{hard} constraints,~and at the~same has the~lowest possible score of \textit{soft} constraints.
Since this is a~problem that is often solved and analyzed in practice,~there are already countless methods used in implementing the~solution.
The~categories into which the~ways of solving optimisation problems can be divided are from the~various areas.
It is often a~method of \textit{simulated annealing},~the~application of \textit{genetic algorithms},~or generally inspired by \textit{nature},~such as optimising \textit{ants colonies} and last but not least \textit{genetic algorithms}.

\subsection{Formal Definition}
An instance of the UCT problem is defined as follows:
\begin{itemize}
    \item A set of $N$ courses, $e = \{ e_1, \dots, e_N\}$
    \item $45$ time-slots
    \item A set of $R$ rooms
    \item A set of $F$ room features
    \item A set of $S$ students
\end{itemize}
where tor each room,~its capacity and the~features it provides are also given. 
For each student,~there is a~list of subjects to be enrolled in,~and for each course,~lists the~required features.
Further, this problem includes $4$ \textit{hard} constraints and $3$ \textit{soft} constraints as follows.
\paragraph{Hard Constraints:}
\begin{itemize}
    \item \textbf{Event Conflict} ($H_1$): no student can be assigned to more than one course at the~same time
    \item \textbf{Room Features} ($H_2$): the~room should satisfy the~features required by the~event
    \item \textbf{Room Capacity} ($H_3$): the~number of students attending the~event should be less than or equal to the~room capacity
    \item \textbf{Room Occupancy} ($H_4$): no more than one event is allowed at a time slot in each room
\end{itemize}
\paragraph{Soft Constraints:}
\begin{itemize}
    \item \textbf{Event in the Last Time Slot} ($S_1$): a~student shall not have to attend a~course that is scheduled in the~last time slot of the~day
    \item \textbf{Two Consecutive Events} ($S_2$): a~student shall not have more than two consecutive events
    \item \textbf{One Event a Day} ($S_3$): a~student shall not have to attend a~single course on a~day
\end{itemize}
Once the~solution of the~UCTP instance is found,~it is first assessed in terms of meeting the~inviolable (\textit{hard}) constraints.
If it does not conflict with any of these hard restrictions,~the~solution is marked as \textit{feasible},~so it is a~valid and usable solution.
Violent (\textit{soft}) constraints are then used to assess and compare the~quality of the actual results.
The~lower the~value of the~conflict with these constraints for the~given solution,~the higher the~quality of the~solution.
For \textit{genetic} algorithms,~this \textit{score} represents the~value of the~fitness function of a~given solution - the \textit{chromosome}.
This fitness function for the problem is defined in the
formula below
\begin{align*}
    \min \sum_{s=1}^{S}{s_1 + s_2 + s_3}
\end{align*}
where $s_1$, $s_2$ and $s_3$ represents the relevant soft constraints score individually for each student.

\section{Differential Evolution Algorithm} \label{sec:method}
The~proposed method is named \textit{Differential Evolution Algorithm}, which is based on the~principle of \textit{genetic algorithms},~but unlike them,~the~key operation is not a~\textit{cross} but a~\textit{mutation}.
The~implemented method can be logically divided into two phases. 
In the~first part,~an~initial population is created,~representing a~set of valid schedules,~and then a~\textit{Differential Evolution Algorithm} is applied. 
An~algorithm that aims to increase the~quality of the~solution while maintaining the~integrity of all inviolable (hard) constraints.

\subsection{Constructive Heuristic} \label{sec:const}
The first step of the~algorithm is creating the~\textit{initial population},~i.e. the~production of a~given number of \textit{chromosomes},~each of which represents a~valid solution of the~specified UCTP instance.
Creating one schedule consists of the~gradual assignment of rooms and part of the schedule to each of the~courses.
To increase the~success of this initial creation,~the~courses are first sorted according to the~number of rooms to which they can be assigned \,--\, which is determined based on the~capacity of the~room and its features.
Subsequently,~the~sorted courses are placed in the~schedule gradually from those to which the~smallest number of rooms can be assigned.
We note,~that in this process,~the~soft restrictions are not taken into account at all in this process.
When a~\textit{feasible} solution can be created in this way,~the current \textit{chromosome} is added to the~\textit{population}, and the~algorithm continues to create a~new solution or by moving to the~second part of the~algorithm.

Otherwise,~when the~algorithm gets into a~situation where there is no free time and room for the~currently placed course,~the~\textit{neighbourhood moves} are applied to the~yet placed courses.
The aim is to rearrange the finished part of the~schedule to free space for the~currently placed course.
First,~the~\textit{neighbourhood move} $N_1$ is performed several times. The algorithm continues by placing the~current course when the~schedule was successfully rearranged.
Otherwise,~the \textit{neighbourhood move} $N_2$ is performed repeatedly.
If it is still not possible to find a~place for the~currently placed course,~the~placement of the~following courses is continued.
The~different neighbourhood structures and their explanation can be outlined as follows:

\paragraph{Neighbourhood Move $N_1$.}
Choose a~single course at random and move to a~feasible time slot that can generate the~lowest penalty cost.

\paragraph{Neighbourhood Move $N_2$.}
Select two courses at random from the~same room (the room is randomly selected) and swap time slots.

\noindent\\
The~technique described above should lead to one valid solution to the~problem. 
When the current schedule cannot be successfully filled, this solution is removed from the~\textit{population}.

\subsection{Improvement Algorithm}
Once the~complete initial population is generated,~the~second phase of the~whole algorithm can begin,~which is the~optimisation by the~\textit{evolutionary algorithm}.
\textit{Genetic algorithms} inspire the implemented differential evolution algorithm,~and its main emphasis is on \textit{mutation} operations.
One iteration of the~described algorithm consists of four parts:~\textit{mutations}, \textit{crossover},~\textit{evaluation},~and \textit{selection}.
This sequence of steps is applied repeatedly until at least one of the~terminating conditions is reached.
These are the~achievement of a~fixed number of iterations (generations) or the~acquisition of a~solution whose value of the~fitness function is equal to 0.
The~base schema of this algorithm is presented below:

\begin{align*}
    & Initialisation \\
    & Evaluation \\
    & \textbf{do while} \; (termination \; criteria \; are \; met) \\
    & \quad Mutation \\
    & \quad Crossover \\
    & \quad Evaluation \\
    & \quad Selection \\
    & \textbf{end do while}
\end{align*}

\begingroup\vspace*{-\baselineskip}
\captionof{figure}{Differential Evolution Algorithm.}
\vspace*{\baselineskip}\endgroup

\paragraph{Mutation.}
The~\textit{mutation} is a central part of the~implemented differential evolution algorithm.
The~first step of the~mutation is the~random selection of two parental chromosomes \,--\, i.e. two complete solutions \,--\, from the~population.
Then a~mutation is performed by selecting one of the~two mutation operators and then applying it to each of the~parents separately.
Mutation operators are the~same as neighbourhood moves described in Section~\ref{sec:const}.
This results in a~modified pair of \textit{parental chromosomes}.

\paragraph{Crossover.}
Crossover is an~operation whose purpose is to generate two new descendants.
The~input of this operation is two-parent chromosomes obtained by mutation operation.
For each parent,~one time is randomly generated in the~schedule,~and in the~parent chromosomes,~the courses planned for a~given time are interchanged.
In order to maintain the~validity of the~solution in the~offspring,~two conditions are placed on the~exchange of the~course.
The~course can be moved only if it is free in the~second schedule on the~moving position, and moving the course to a~new location will not cause any collisions between courses.

\paragraph{Evaluation and Selection.}
Evaluation is used to evaluate newly born offspring.
To evaluate each of the~offspring,~a \textit{fitness function} is used,~which calculates a~penalty for non-compliance with violable (\textit{soft}) restrictions.
If a~better offspring has a~lower value of \textit{fitness function} than the~best individual in the~population so far,~the worst-rated chromosome in the~population is replaced by the~better of the~offspring.

\section{Implementation} \label{sec:implementation}
The described algorithm has been implemented in Python programming language.
At least,~it is required Python version 3.6.
Python libraries NumPy\footnote{Python library for mathematics computation
NumPy\,--\,\url{https://numpy.org}.} and SciPy\footnote{Python library
for mathematics computation SciPy\,--\,\url{https://www.scipy.org}.} was used for mathematics computation and work with series and lists.
Required versions of these libraries,~as well as the~next needed libraries,~are specified in the~file \texttt{requirements.txt}.
The~implementation consists of several modules described in the~paragraphs below.

\noindent \\
The module \texttt{ucctp.py} severs an~entry point of the application.
It sets a~routine for logging,~parses arguments from the~command line and calls the~main function of the~whole application.
The module \texttt{input.py} reads and validates an input file with the specification of the problem.
It contains functions for reading and parsing input files with the~specification of the~problem,~and function for writing solutions to output file in the~appropriate format.

\noindent\\
A~class \texttt{UcctpInstance} represents the~University Course Timetabling Problem.
This class encapsulates the~specification of the~problem as well as its input parameters.
Moreover, it contains operations over the problem instance that cannot be derived directly from input data.
For instance,~there is implemented the~method for sorting the~events according to the~number of rooms which they can be assigned,~the~method for computing the~collision events,~so events that have one or more common students,~or the~method to compute available rooms for courses,~and some other auxiliary methods.

\noindent\\
The most interesting is module \texttt{differential\_e\-volution.py} which contains main functions.
The method \texttt{run} serves as the~wrapper over the~whole algorithm,~and it gradually performs all required operations.
Firstly,~it loads and parses the algorithm parameters from the~file \texttt{parameters.json} and save them into the~relevant program structure.
The~next step is to create an~initial population of the chromosomes. 

This is done using the~\texttt{constructive\_heuri\-stics} method,~which creates a~population of the required size by repeatedly calling the~\texttt{construct\_e\-vent\_time\-table} method.
In the~way described above tries to place all the~courses one after the~other and thus create a~complete schedule. 
If there is nowhere to place a~course,~the already mentioned neighbourhood moves are applied. 
In particular,~the~\texttt{apply\_mo\-ve\_n1} and \texttt{apply\_move\_n2} methods are used for this purpose.
The~chromosomes that are not complete,~i.e. even after the~application of neighbourhood moves,~no suitable location was found for any course,~are not appended to the~population.
Moreover,~as the~fulfilment of inviolable constraints is che\-cked already during the~creation of the~schedule,~we do not have to incorrectly remove chromosomes after generation.

Subsequently,~the~\texttt{evaluate\_timetables} me\-thod calculates the~evaluation of all chromosomes in a~population-based on how much they do not satisfy the~violation (soft) constraints. 
The \texttt{object\_func\-tion} method is implemented to calculate the penalty for violating restrictions $s_1$, $s_2$ and $s_3$.
The~quality of each individual is then given as the~sum of these three values. 
According to the~evaluation of individual chromosomes,~the~entire population is sorted from the~highest quality to the~lowest quality in the~same method (\texttt{evaluate\_timetables}).

Now the~differential evolution algorithm itself can begin. 
Until the~specified number of iterations is exceeded,~or the~best solution in the~population does not have a~zero value of the~fitness function,~the \texttt{differe\-ntial\_evolution\_algorithm} method is repeatedly called,~which performs one iteration implemented algorithm.
Thus,~the~\texttt{mutation} and \texttt{crossover} methods are gradually performed. 
Then,~the~evaluation is performed,~and if one of the~offspring represents a~better solution than the best one so far,~the~new offspring replaces the~worst solution in the~population.

\subsection{Algorithm Enhancement}
The implementation of the~original algorithm often had a~problem finding a~\textit{feasible} solution.
Moreover,~the~paper does not contain a~strictly detailed description of how the~creation of the~original population should take place.
As a~result,~there was space in this area for change and improved the~existing algorithm.

The~placement of the~course is still gradual from the~most problematic one.
For the~selection of the room and the~time slot until which the~course will be filled,~we implemented the~\texttt{get\_timetable} method, which selects the~most suitable of all available locations.
In more precise,~the~method \texttt{get\_final\_ti\-metable} first selects the~most suitable time slots,~wh\-ich are the~ones in which there are already the~most occupied rooms.
The~reason is that such time slots will be the~hardest to fill in the~future,~as there is a~high probability that a~conflict rate will already be located here.
Therefore,~all possible course locations are limited to those that take place in the~most appropriate time slots selected.

Subsequently,~from the~remaining locations,~the~on\-es that take place in the~most suitable rooms are selected by the~\texttt{get\_final\_time\-table} method.
Th\-ese are the~rooms in which the~smallest number of courses that have not yet been filled can be placed. 
Here the reason is clear,~again the~aim is to provide as many opportunities as possible for further courses.
From these locations that meet the~above conditions for both the~room and the~time slot is then randomly selected as the~final location.

\section{Experimental Evaluation} \label{sec:experiments}
The~reference to the~set of tests on which the~original implementation was tested is no longer functional and could not be traced on other servers.
However,~we found the GitHub repository\footnote{\url{https://github.com/tomas-muller/cpsolver-itc2007}} which contains the International Timetabling Competition data-sets.
From this data set,~we took a~set of instances with a~similar structure to the~one in the~original paper.
As well as the~original set,~the~new one contains a~total of 11 tests,~of which 5 correspond to the~specification of small test input,~the~other 5 are medium and 1 input problem falls into the~category of large inputs.
A~description of the~properties that the~input file of a~given size meets is given in the~following table:
\begin{table}[ht!]
\centering
\begin{tabular}{|l|c|c|c|}
\hline
\textbf{Category} & \multicolumn{1}{l|}{\textbf{Small}} & \multicolumn{1}{l|}{\textbf{Medium}} & \multicolumn{1}{l|}{\textbf{Large}} \\ \hline
\textbf{N (courses)}        & 100                                 & 300-400                              & 400                                 \\ \hline
\textbf{R (rooms)}        & 5                                   & 10                                   & 10                                  \\ \hline
\textbf{F (features)}        & 5                                   & 5                                    & 10                                  \\ \hline
\textbf{S (students)}        & 80                                  & 200                                  & 200                                 \\ \hline
\textbf{max N per S}  & 20                                  & 20                                   & 20                                  \\ \hline
\textbf{max S per N}  & 20                                  & 50                                   & 100                                 \\ \hline
\textbf{max F per R}  & 3                                   & 3                                    & 5                                   \\ \hline
\textbf{\% usage of F}     & 70                                  & 80                                   & 90                                  \\ \hline
\end{tabular}
\caption{Description of the properties of individual categories of test inputs.}
\end{table}

To mimic the~experiments performed in the~original paper as much as possible,~all arguments were set to the~same values when experimenting as in the~original experiments performed by the~paper authors.
An overview of the parameters values used is as follows:
\begin{lstlisting}[language=json,firstnumber=1]
{
    "generations_number": 200000,
    "population_size": 50,
    "crossover_rate": 0.8,
    "mutation_rate": 0.5,
    "n1_applications": 100,
    "n2_applications": 100
}
\end{lstlisting}

\begin{table*}[ht!]
\begin{center}
\begin{tabular}{|l|c|c|c|c|c|c|}
\hline
\textbf{Input}   & \multicolumn{1}{l|}{\textbf{Paper}} & \multicolumn{1}{l|}{\textbf{xvitra00}} & \multicolumn{1}{l|}{\textbf{xmarus06}} & \multicolumn{1}{l|}{\textbf{xvales03}} & \multicolumn{1}{l|}{\textbf{xvales02}} & \multicolumn{1}{l|}{\textbf{xstupi00}} \\ \hline
\textbf{small1}  & 0                                   & \textbf{11}                            & 33                                     & 31.2                                   & 47                                     & 42                                     \\ \hline
\textbf{small2}  & 0                                   & 21                                     & 32                                     & 34.5                                   & 21                                     & \textbf{20}                            \\ \hline
\textbf{small3}  & 0                                   & \textbf{16}                            & 26                                     & 28.0                                   & 38                                     & 25                                     \\ \hline
\textbf{small4}  & 0                                   & 45                                     & 42                                     & \textbf{41.7}                          & 53                                     & 46                                     \\ \hline
\textbf{small5}  & 0                                   & \textbf{20}                            & 30                                     & 32.5                                   & 37                                     & 27                                     \\ \hline
\textbf{medium1} & 160                                 & \textbf{27}                            & 234                                    & 255.9                                  & 237                                    & 198                                    \\ \hline
\textbf{medium2} & 81                                  & 154                                    & 132                                    & 126.2                                  & \textbf{62}                            & 71                                     \\ \hline
\textbf{medium3} & 149                                 & \textbf{117}                           & 269                                    & 289.8                                  & 205                                    & 230                                    \\ \hline
\textbf{medium4} & 113                                 & \textbf{26}                            & 170                                    & 173.1                                  & 66                                     & 51                                     \\ \hline
\textbf{medium5} & 143                                 & \textbf{27}                            & 145                                    & 168.2                                  & 76                                     & 59                                     \\ \hline
\textbf{large}   & 735                                 & 369                                    & 468                                    & 408.4                                  & 342                                    & \textit{\textbf{325}}                  \\ \hline
\end{tabular}
\caption{Comparison of various algorithms on the individual input instances.}
\label{tab:compare}
\end{center}
\end{table*}

\subsection{Results Comparison}
As in the~original experiments,~a~total of 11 times were tested for each test problem,~and the~lowest value of the~result was always selected for this report.
Due to the absence of the~original set,~the~results of the~implementation were compared with the~implementations of the~algorithms implemented by the students in the previous years, which are available in their GitHub pages\footnote{\url{https://github.com/marusak/UCTPS/blob/master/doc.pdf}}\footnote{\url{https://github.com/Bihanojko/SNT/blob/master/doc.pdf}}.
A~comparison of the algorithms on the~
individual test inputs is given in Table~\ref{tab:compare}.
For each input instance of the~problem,~the~best of the~achieved values is marked here in bold,~i.e.~the~algorithm that achieved the~best solution on the~given input.
The~implemented algorithm achieved two times the~best result. 
On the~contrary,~only in one case,~it turned out the~worst at all,~so it is a~competitive algorithm.
We note, that the~\textit{xvales02} implements the~same method as we.
Compared with the~original implementation,~Table~\ref{tab:compare} also shows the~values obtained by the authors of the original paper. 
However,~this comparison is not very meaningful,~as the~input instances correspond in size,~but still be one-sided problems and values of the~best possible solutions can vary considerably.

\subsection{Influence of population-size parameter}
The~next part of the~experimenting deals with examining the~influence of the~value of the~\textit{population-size} parameter on the~population's quality.
This parameter certainly affects the~value of the~\textit{fitness function} of the~solution,~the~larger the~original population,~the~more parent chromosomes can be selected for crossbreeding,~and thus the~more diverse chromosomes we can obtain.
For each input set,~experiments were performed with a~\textit{population size} ranging from 10 to 100 individuals with a~step of 10.
The~number of generations was always set to 100 so that the~iterative increase in population quality was not reflected too much in the~results of the~experiment.
The~resulting characteristics for each of the 11 test inputs are shown in Figures~\ref{fig:smallp},~\ref{fig:mediump},~\ref{fig:largep}.
From the ~graphs,~it is clear that randomness greatly influences the~quality of the~population. 
Still,~when looking at the~whole curve,~the~expected hypothesis is confirmed: The~greater the~number of individuals in the~population,~the higher the~quality of the~population.

\subsection{Influence of generations-number parameter}
The~last part of the~experiments is devoted to examining the effect of the~\textit{generations-count} parameter on the~quality of the~resulting solution.
This parameter significantly affects the~value of the~\textit{fitness function} of the~result,~as the~population can improve in each generation.
Its research can also be useful if we would like to use the~application for quality calculations in the~shortest possible time because,~for a~higher number of generations,~the~calculation time starts to increase significantly.
For each input set,~the~experiment has performed 10 times,~and then the~run that achieved the~greatest improvement over the~generations was plotted.
The~experiments were performed with a~population size of 50 individuals,~and the~quality of the~population was recorded in the~range from 0 to 20000 generations with step 1. 
The~values of the~graph were obtained in an~excerpt from the application after each iteration.
The~resulting characteristics for each of the~11 test inputs are shown in Figure~\ref{fig:smallg},~\ref{fig:mediumg},~\ref{fig:largeg}. 
There is a~clear improvement in all cases,~with a~near logarithmic decrease at the~beginning of the~curve,~and later primarily a~linear.

\noindent \\
These~experiments confirmed both the~correctness of the~re-implementation of the~algorithm and the~functionality and efficiency of the~original algorithm.

\begin{figure*}[h]
\centering
\includegraphics[height=0.30\textheight]{figures/small_population.pdf}
\caption{Influence of population-size parameter for small problem instances.}%
\label{fig:smallp}%
\end{figure*}

\begin{figure*}[h]
\centering
\includegraphics[height=0.30\textheight]{figures/medium_population.pdf}
\caption{Influence of population-size parameter for medium problem instances.}%
\label{fig:mediump}%
\end{figure*}

\begin{figure*}[h]
\centering
\includegraphics[height=0.30\textheight]{figures/large_population.pdf}
\caption{Influence of population-size parameter for large problem instance.}%
\label{fig:largep}%
\end{figure*}

\begin{figure*}[h]
\centering
\includegraphics[height=0.30\textheight]{figures/small_generation.pdf}
\caption{Influence of generations-number parameter for small problem instances.}%
\label{fig:smallg}%
\end{figure*}

\begin{figure*}[h]
\centering
\includegraphics[height=0.30\textheight]{figures/medium_generation.pdf}
\caption{Influence of generations-number parameter for medium problem instances.}%
\label{fig:mediumg}%
\end{figure*}

\begin{figure*}[h]
\centering
\includegraphics[height=0.30\textheight]{figures/large_generation.pdf}
\caption{Influence of generations-number parameter for large problem instance.}%
\label{fig:largeg}%
\end{figure*}

\section{Conclusion} \label{sec:conclusion}

As part of this project,~an~algorithm called \textit{Differential Evolution Algorithm} was studied and re-implemented.
Subsequently,~several experiments were performed with the~implementation,~comparing this approach with other algorithms,~examining the~effect of population size on the~quality of solutions,~and the~number of generations on the~quality of the~total population.
The~functionality of the~re-implementation was verified by these experiments.
\fi

%--------------------------------------------------------
%--------------------------------------------------------
%--------------------------------------------------------
%	LITERATURA
%--------------------------------------------------------
%--------------------------------------------------------
\phantomsection
% \ifczech
%     \bibliographystyle{bib-styles/czplain}
% \fi
% \ifslovak
%     \bibliographystyle{bib-styles/skplain}
% \fi
\ifenglish
    \bibliographystyle{bib-styles/enplain}
\fi

\bibliography{2019-PPFIT-ShortName-bib}

%--------------------------------------------------------
%--------------------------------------------------------
%--------------------------------------------------------
\end{document}